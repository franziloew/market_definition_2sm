\documentclass[10pt,a4paper]{scrreprt}
\usepackage[utf8]{inputenc}
\usepackage[english]{babel}
\usepackage{amsmath}
\usepackage{amsfonts}
\usepackage{amssymb}
\usepackage{pgfplots}
\usepackage{caption}
\usepackage{graphicx}
\usepackage{tikz-3dplot}
\usepackage{subcaption}
\usepackage{float}
\usepackage{multirow,rotating}

\usepackage[autostyle]{csquotes}

\usepackage[backend=biber,style=authoryear-comp]{biblatex}
\addbibresource{Meine Bibliothek.bib}


\begin{document}
\chapter{Introduction}

This paper is related to a relatively recent line of economic literature, investigating the implications of two-sided markets on competition policy and offering different approaches to deal with the feedback effects between demand on multiple market sides. While the first policy contributions mainly criticized the application of standard policy to those markets (\cite{wright_one-sided_2004}; \cite{leonello_horizontal_2010}; \cite{chandra_mergers_2009}), more recent work has also intended to suggest alternative approaches (\cite{argentesi_estimating_2007}; \cite{song_estimating_2015}). We try to contribute to the latter by offering a new approach to define a two-sided market. 

The literature of two-sided markets was pioneered by the theoretical work of \cite{caillaud_chicken_2003}, \cite{rochet_platform_2003}, \cite{evans_antitrust_2003} and \cite{armstrong_competition_2006}, whereby the definition given by \cite{evans_antitrust_2003} can be seen as a particular case of the more general definition proposed by \cite{rochet_platform_2003} (\cite{filistrucchi_identifying_2012}). \cite{rochet_platform_2003} as well as \cite{armstrong_competition_2006} both provide a theoretical concept to analyze how platforms chose prices in a market with two consumer sides (networks) showing indirect network effects. However, there are a number of modeling differences between the two articles with regard to (a) the platform's cost structure, (b) the fee the consumers on both market sides have to pay and (c) the source of consumer heterogeneity. For most of their analysis \cite{rochet_platform_2003} assume that the platform incurs in a per-transaction cost and charges a usage fee, whereas \cite{armstrong_competition_2006} considers membership fees and per-agent cost.\footnote{\cite{rochet_platform_2003} as well as \cite{armstrong_competition_2006} both consider the case where there are fixed fees as well as per-transaction fees for model where consumers can only single-home.} With respect to the source of consumer heterogeneity (c) in \cite{rochet_platform_2003} consumers are heterogenous in the benefit they get from the interaction with the respective other market side or, in other words, the indirect network effect varies between the consumers. In \cite{armstrong_competition_2006} the indirect network effect only differs between the market side and is homogenous among agents on the same side. Heterogeneity is given by differences in consumers' membership values.\footnote{A more detailed discussion of these assumptions with regard to our approach is provided in \ref{model}} For the monopoly case the equilibrium prices of a profit-maximizing platform on one market side is given by the cost of providing the service, adjusted downwards by the magnitude of the indirect network effect and adjusted upwards by the elasticity of demand on that side (\cite{armstrong_competition_2006}). In the model of \cite{rochet_platform_2003} the price level charged by a profit-maximizing platform will be given the classical Lerner formula, where elasticity is the sum of the two elasticities in each side.  
 
In a more recent paper \cite{rochet_two-sided_2006} provide an analysis of the monopoly case, where agents' are heterogeneous both regarding the indirect network effect and the membership value. However, \cite{weyl_price_2010} generalizes the model in \cite{rochet_two-sided_2006} as he allows agents to be heterogenous in the type of the indirect network effects: The membership externality, occurring when an additional membership has a positive effect on the other market side, and usage externality, when the benefit is originated by an additional transaction. To avoid the equilibrium multiplicity and to overcome the coordination problem that maybe faced by a platform in having both sides "on board",\footnote{This coordination problem is also known as chicken $\&$ egg problem (\cite{caillaud_chicken_2003})} \cite{weyl_price_2010} assumes, that the platform directly choses the participation level on both sides, rather then the price structure.\footnote{He refers to this as insulated tariffs (See also \cite{white_insulated_2012})}

To identify a market as being two-sided, the interconnection of the two market sides has to be detected. Most of the literature related to the quantification of the indirect network effects have based their analysis on electronic payments system industries (\cite{ackerberg_quantifying_2006}; \cite{rysman_empirical_2007}) or magazine and newspaper industries (\cite{kaiser_price_2006}, \cite{argentesi_estimating_2007}). \cite{rysman_competition_2004} estimates a structural model of two-sided markets to measure indirect network effects in the market for Yellow Pages. In his setting the platform maximizes profits in choosing the price only on the advertiser side, as reader get the service for free. In \cite{argentesi_estimating_2007}, instead publishers’ profits are the sum of advertising profits and profits from circulation. They use data on the Italian newspaper industry to offer an alternative approach to test for collusion using a structural econometric model characterized by only one indirect network effect from reader to advertiser. \cite{argentesi_market_2005} provide a generalized framework of \cite{argentesi_estimating_2007} with indirect network effects on both market sides (See also \cite{filistrucchi_merger_2010}). Whereas no effect of advertising on the number of readers was found in the daily newspaper market in the US (\cite{fan_ownership_2013}) and in the Belgian daily newspaper market (\cite{cayseele_prices_2009}), \cite{kaiser_price_2006} find that advertising increases readers demand for magazines in Germany. They use an adopted version of the model proposed in \cite{armstrong_competition_2006}. However, \cite{wilbur_two-sided_2008} found a negative effect of advertising on viewers in the television market. His main conclusions are that viewers tend to be adverse to advertising, that advertiser preferences influence network choices more strongly than viewer preferences, and that advertisement avoidance tends to increase equilibrium advertising quantities and decrease network revenues.

As mentioned above, earlier policy contributions criticize the application of standard competition policy on markets that exhibit at least one indirect network effect. \cite{evans_antitrust_2003}, \cite{evans_industrial_2007} \cite{wright_one-sided_2004} and \cite{kaiser_price_2006} are prominent examples of papers that have \textit{FOCUS}ed on competition policy on two-sided markets. They have pointed out that in the presence of indirect network externalities the efficient price structure does not reflect the ratio of marginal cost, nor does increased competition necessarily leads to a more efficient market outcome or merger leads to increased prices.\footnote{\cite{malam_mergers_2011} uses an oligopoly model of competition with differentiated products (based on the approach of \cite{salop_monopolistic_1979}) where ad-sponsored media platforms charge a zero price to viewers when competing simultaneously for advertisers. He shows, that mergers among ad-sponsored platforms have a competition-intensifying effect, which offsets the incentive to increase prices on the advertiser side.} They show that relying on conventional methods to analyze mergers in two-sided markets will lead to significantly different results than using methods that explicitly incorporate the two-sided nature of this markets. \cite{evans_antitrust_2003} argues, that defining a relevant market for antitrust purposes looking at only one side can lead to a market definition which is too narrow. In a more recent study \cite{evans_analysis_2008} analyze the Google and DoubleClick case, confirming, that the Lerner pricing formula does not hold for two-sided markets. While predatory pricing is a practice that harms competition in case of traditional industries\footnote{Industries with only one market side.}, selling a product below marginal cost\footnote{Or even for free, as is the case for the search-engine market as well as many digital markets.} can be a profit maximizing strategy rather than an attempt to predate in a two-sided market (\cite{wright_one-sided_2004}). \cite{wright_one-sided_2004} also argues, that increased competition does not necessarily lead to more efficient prices from the social point of view. A analysis of the Canadian newspaper industry shows, that mergers in two-sided markets may not necessarily lead to higher prices for either side of the market. Even a merger to monopoly might raise welfare and do so even in the absence of efficiency gains (\cite{leonello_horizontal_2010}). These papers emphasizes  the need for alternative approaches to adopt competition policy that adequately hits the requirements of two-sided markets. 

\cite{filistrucchi_ssnip_2008} discuses the application of a modified SSNIP test in order to determine the relevant two-sided market. He suggests a distinction of the two-sided markets regarding the observability of transaction costs.\footnote{Whereas \cite{filistrucchi_ssnip_2008} uses the terms “media type” and “payment card type”, \cite{filistrucchi_market_2013} use the terms “non-transaction” and “transaction” marktes.} In the "payment card type" market the platform can observe the transaction cost between the two market sides, whereas in the "media type" market the transaction cost does not exist (or is not observable to the platform, e.g. reader reads an ad). In \cite{filistrucchi_market_2013} the authors point out, that in two-sided non transaction markets, two (interrelated) markets need to be defined, while in transaction markets, only one market side should be defined.\footnote{We will discuss this suggestion as well as the application of a modified SSNIP test in the upcoming chapter \ref{SSNIP}.} \cite{emch_market_2006} and \cite{alexandrov_antitrust_2011} show how a SSNIP test should be performed in a two-sided non transaction market. \cite{white_insulated_2012} present a UPP formulae for two-sided markets assuming that firms charge insulating tariffs, meaning that platforms choose quantities and then support those quantities by the corresponding insulating tariffs and \cite{noel_analyzing_2005} suggests an extension of the Critical Loss Analysis as an alternative method to define two-sided markets.\footnote{See \cite{evans_two-sided_2012} and \cite{filistrucchi_identifying_2012} for a discussion of market definition in two-sided markets.} Beside market definition, merger simulations are of major interest with regard to policy implications of two-sided markets. \cite{evans_analysis_2008} argue that standard merger tools are biased in two-sided markets and offer an extension applicable for two-sided markets. They illustrate their techniques with an application to an acquisition involving the multisided online advertising industry. \cite{fan_ownership_2013}; \cite{filistrucchi_merger_2010}; \cite{filistrucchi_assessing_2012} and \cite{jeziorski_merger_2010} propose different structural econometric models to perform merger simulation in different two-sided markets such as newspapers and radio. \cite{filistrucchi_merger_2010} use a structural econometric framework to simulate the effects of mergers among two-sided platforms selling differentiated products and competing á la Bertrand. They extend the supply model of \cite{argentesi_estimating_2007} to the case of a two-sided market with two indirect network effects. Using a similar approach \cite{filistrucchi_assessing_2012} compare different methods to assess unilateral merger effects in a two-sided market by applying them to a hypothetical merger in dutch newspaper industry, where consumers on both sides pay a price to access the platform. This paper contributes to the body of research that provides practical suggestions to practitioners. We use data on quantity to analyze substitutional effects on two-sided markets. The advantage of using quantity data is clear: As price levels and price structure in two sided markets are closely linked to the scope of indirect network effects, they can hardly be analyzed in the conventional way of antitrust economics. 
 



\chapter{SSNIP and Two-Sided Markets}\label{SSNIP}

\end{document}
